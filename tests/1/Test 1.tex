\documentclass[12pt, letterpaper]{article}
\usepackage[english, russian]{babel}
\usepackage{amsmath} % for brackets
%\pagestyle{empty} % off page numbers

%opening
\title{
	Контрольная работа №1. \\
	Теория полезности. Модель индивидуальных рисков
	}
\date{}

\begin{document}

\maketitle

\newpage

\begin{center}
	\section{Вариант}
\end{center}

% task 1
\begin{enumerate}
	\item 

	Страхователь подвержен случайным потерям $X \sim Uniform(0, 10)$.
	Вычислить максимальную премию, которую готов заплатить страхователь с капиталом $w=10$ и имеющий функцию полезности $U(w) = \sqrt{x}$ за полное страхование.


% task 2
\item
	Вероятность наступления страхового случая по договору равна $0.03$. В этом случае случайные выплаты имеют плотность
	\[
		f(x) = 
		\begin{cases}
			3x^2, & x \in [0, 1],\\
			0, & \text{иначе}
		\end{cases}
	\]
	\begin{itemize}
		\item Найти среднее и дисперсию суммапрных выплат для портфеля из 100 таких же договоров.
		\item Используя нормальную аппроксимацию, оценить вероятность того, что суммарные выплаты привысят средние более чем на $5\%$
	\end{itemize}
	
% task 3
\item
	Случайная величина $X_1$ имеет экспоненциальное распределение с параметром 3, а $X_2$ - экспоненциальное, с параметром $1$.
	Предполагая независимость случайных величин, найти плотность распределения $S = X_1 + X_2$

\end{enumerate}

\newpage
\begin{center}
	\section{Вариант}
\end{center}

% task 1
\begin{enumerate}
	\item 
	

	Страхователь подвержен случайным потерям $X \sim Exp(3)$.
	Вычислить максимальную премию, которую готов заплатить страхователь с капиталом $w=30$ и имеющий функцию полезности $U(w) = e^{-0.1}$ за полное страхование.
	% task 2
	\item
	Вероятность наступления страхового случая по договору равна $0.01$. В этом случае случайные выплаты имеют плотность
	\[
	f(x) = 
	\begin{cases}
		5x^4, & x \in [0, 1],\\
		0, & \text{иначе}
	\end{cases}
	\]
	\begin{itemize}
		\item Найти среднее и дисперсию суммарных выплат для портфеля из 50 таких же договоров.
		\item Используя нормальную аппроксимацию, оценить вероятность того, что суммарные выплаты привысят средние более чем на $1\%$
	\end{itemize}
	
	% task 3
	\item
	Случайная величина $X_1$ имеет экспоненциальное распределение с параметром 2, а $X_2$ - равномерное на отрезке $[0, 2]$.
	Предполагая независимость случайных величин, найти плотность распределения $S = 4X_1 + X_2$
\end{enumerate}


\newpage
\begin{center}
	\section{Вариант}
\end{center}

% task 1
\begin{enumerate}
	\item 
	
	Страхователь подвержен случайным потерям $X \sim Exp(1)$.
	Вычислить максимальную премию, которую готов заплатить страхователь с капиталом $w=10$ и имеющий функцию полезности $U(w) = e^{-0.01}$ за полное страхование.
	
	
	% task 2
	\item
	Вероятность наступления страхового случая по договору равна $0.01$. В этом случае случайные выплаты имеют функцию распределения
	\[
	F(x) = 
	\begin{cases}
		0, & x < 0\\
		x^6, & x \in [0, 1],\\
		1, & x > 1
	\end{cases}
	\]
	\begin{itemize}
		\item Найти среднее и дисперсию суммапрных выплат для портфеля из 150 таких же договоров.
		\item Используя нормальную аппроксимацию, оценить вероятность того, что суммарные выплаты не привысят средние более чем на $5\%$
	\end{itemize}
	
	% task 3
	\item
	Случайная величина $X_1$ имеет экспоненциальное распределение с параметром 2, а $X_2$ - экспоненциальное, с параметром $5$.
	Предполагая независимость случайных величин, найти плотность распределения $S = 7X_1 + X_2$
\end{enumerate}


\newpage
\begin{center}
	\section{Вариант}
\end{center}


\begin{enumerate}
	\item 	
	Страхователь подвержен случайным потерям $X \sim Uniform(0, 5)$.
	Вычислить максимальную премию, которую готов заплатить страхователь с капиталом $w=5$ и имеющий функцию полезности $U(w) = \sqrt{x}$ за полное страхование.

	\item
	Вероятность наступления страхового случая по договору равна $0.01$. В этом случае случайные выплаты имеют функцию распределения
	\[
	F(x) = 
	\begin{cases}
		0, & x < 0\\
		\frac{x}{8}, & x \in [0, 8],\\
		1, & x > 8
	\end{cases}
	\]
	\begin{itemize}
		\item Найти среднее и дисперсию суммарных выплат для портфеля из 150 таких же договоров.
		\item Используя нормальную аппроксимацию, оценить вероятность того, что суммарные выплаты не превысят средние более чем на $5\%$
	\end{itemize}
	

	\item
	Случайная величина $X_1$ имеет экспоненциальное распределение с параметром 1, а $X_2$ - равномерное на отрезке $[0, 1]$.
	Предполагая независимость случайных величин, найти плотность распределения $S = X_1 + 3X_2$
\end{enumerate}


\end{document}
