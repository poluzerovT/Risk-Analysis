\documentclass[14pt, a4paper]{article}
\usepackage[english, russian]{babel}
\pagestyle{empty} % off page numbers

%opening
\title{
	Контрольная работа №2. \\
	Модели коллективных рисков
}
\date{}

\begin{document}
	%\maketitle
	
	
	\newpage
	\begin{center}
		\section{Вариант}
	\end{center}
	
	\begin{enumerate}
		\item 
		Совокупные выплаты имеют сложное пуассоновское распределение с параметром $\lambda = 1$ и распределением отдельных выплат $P(X=1) = 1 - P(X=3) = 0.8$.
		
		Вычислить вероятности $P(S=k), k = 0, ..., 6$ основным методом.
		
		\item 
		В модели коллективного риска количество выплат имеет биномиальное распределение с параметрами $n=3, p=0.1$.
		
		Найти средние совокупные выплаты, их дисперсию и производящую функцию моментов, если отдельные выплаты имеют распределение:
		
		\begin{center}
			\begin{tabular}{ |c|c|c|c|c|}
				\hline
				$X$ & 1 & 2 & 3 & 4 \\
				\hline
				$P(X)$ & 0.1 & 0.2 & 0.5 & 0.2 \\
				\hline
			\end{tabular}
		\end{center}
	\end{enumerate}


	\newpage
	\begin{center}
		\section{Вариант}
	\end{center}
	
	\begin{enumerate}
		\item 
		Совокупные выплаты имеют сложное пуассоновское распределение с параметром $\lambda = 2$ и распределением отдельных выплат $P(X=1) = 1 - P(X=3) = 0.4$.
		
		Вычислить вероятности $P(S=k), k = 0, ..., 6$ методом, основанным на свойстве перегруппированной суммы.
		
		\item 
		В модели коллективного риска количество выплат имеет биномиальное распределение с параметрами $n=3, p=0.4$.
		
		Найти средние совокупные выплаты, их дисперсию и производящую функцию моментов, если отдельные выплаты имеют распределение:
		
		\begin{center}
			\begin{tabular}{ |c|c|c|c|c|}
				\hline
				$X$ & 1 & 2 & 3 & 4 \\
				\hline
				$P(X)$ & 0.5 & 0.2 & 0.1 & 0.2 \\
				\hline
			\end{tabular}
		\end{center}
	\end{enumerate}


	\newpage
	\begin{center}
		\section{Вариант}
	\end{center}
	
	\begin{enumerate}
		\item 
		Совокупные выплаты имеют сложное пуассоновское распределение с параметром $\lambda = 3$ и распределением отдельных выплат $P(X=2) = 1 - P(X=3) = 0.5$.
		
		Вычислить вероятности $P(S=k), k = 0, ..., 6$ рекуррентным методом.
		
		\item 
		В модели коллективного риска количество выплат имеет биномиальное распределение с параметрами $n=3, p=0.5$.
		
		Найти средние совокупные выплаты, их дисперсию и производящую функцию моментов, если отдельные выплаты имеют распределение:
		
		\begin{center}
			\begin{tabular}{ |c|c|c|c|c|}
				\hline
				$X$ & 1 & 2 & 3 & 4 \\
				\hline
				$P(X)$ & 0.2 & 0.3 & 0.2 & 0.3 \\
				\hline
			\end{tabular}
		\end{center}
	\end{enumerate}
\end{document}
