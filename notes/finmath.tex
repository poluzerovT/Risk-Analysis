\documentclass{article}

\usepackage[main=russian,english]{babel}

\newcounter{example}[section]
\newenvironment{example}[1][]{\refstepcounter{example}\par\medskip
	\noindent \textbf{Задание~\theexample. #1} \rmfamily}{}

% insert images
\usepackage{wrapfig}
\usepackage{graphicx}

% math
\newtheorem{rustheorem}{Теорема}
\usepackage{amsmath}

\begin{document}
	\section{Начисление процентов на счёт}
	
	\subsection{Простые проценты}
	
	Если сумма $C$ помещается на счёт, который платит величину $Ci$ за каждый год жизни счёта, то такое начисление процентов называется простым по ставке $i$ в год. 
	
	На счёте через $n$ лет будет лежать сумма 
	\[
	C(1 + in)
	\]
	Эта величина $C(1 + in) = C + I$ состоит из исходного капитала $C$ и начисленных процентов $I = Cin$
	
	\textit{Замечание}. Если $n$ - не целое число, то формулы справедливы если предположить пропорциональное начисление процентов внутри неполного года.
	
	\begin{example}
		Пусть 860 вкладывается на счет под ставку $5 \frac{3}{8} \%$ в год. Найти сумму, которая будет на счёте спустя 1)6 месяцев, 2)10 месяцев, 3)1 год.	
	\end{example}
	
	\begin{example}
		Вычислить за какое время $n$ сумма на счёте удвоится при начислении простых процентов по ставке $i=14\%$
	\end{example}
	
	 
	\subsection{Сложные проценты}
	Если в конце каждого года начисленные проценты присоединяются к общей сумме на счете, и начисление следующих процентов идет на эту увеличенную сумму, то говорят о начислении сложных процентов.
	
	Пусть $A_n$ - сумма на счете спустя $n$ лет. $A_0 = C$
	\[
	A_{n + 1} := A_{n} + i A_{n} = A_{n} (1 + i)
	\]
	\[
	 A_{n} = C (1 + i) (1+ i) \cdot ... \cdot (1+i) = C(1 + i)^{n}
	\]
	
	При расчете стоимостей финансовых интсрументов, инвестиционных проектов итд., используется идея сложного начисления процентов.
	
	\begin{example}
		Сумма 100 кладется на счёт. Для случаев простого и сложного накопления процентов, построить таблицу, в которой приводятся накопленные сумы через 6 месяцев, 1 год, 5, 10, 20 лет при 1) $i=4\%$, 2) $i=8\%$.  
	\end{example}
	
	\begin{example}
		 Вычислить за какое время $n$ сумма на счёте удвоится при начислении сложных процентов по ставке $i=14\%$
	\end{example}
	
	\begin{example}
		Сколько составляет ежемесячный процентный доход от вклада под $14\%$ годовых на 1)первом году, 2) втором году, 3) пятом году.
	\end{example}
	
	\begin{example}
		Существует две инвестиционные стратегии. Первая: вклад под простой процент по ставке $i_1 = 20\%$. Вторая: вклад под сложный процент по ставке $i_2 = 14\%$. 
		\begin{enumerate}
			\item Изучить зависимость предпочтительности стратегий от срока инвестирования $t$,
			\item При каком $t$ процентные доходы будут отличаться в 2 раза.
		\end{enumerate}
		
	\end{example}
	
	%\newpage	 
	%\section{Процентные ставки}
	
	%\subsection{Номинальные процентные ставки}
	
	
	
\end{document}